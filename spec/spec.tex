
\documentclass[12pt]{spec}

\begin{document}

\specheader
\bigskip
\hrule

\spectitle{Universal Complaint Format Specification}{VERSION: v0.1 | Author: Prathu Baronia}

\section{Overview}

\specjust{A standardized open-source TOML-based format for registering generic
complaints against public or private entities. This format
is \textbf{citizen-facing} and focuses on interoperability across different
complaint systems, departments, and municipalities.}

\specnote{It is \textbf{not} designed for internal workflow or tracking by public departments.}


\section{General structure}

A \textit{.ucf} file is a valid TOML document in `UTF-8` encoding, containing
multiple table sections. Each section describes a logical part of the
complaint.

Each complaint corresponds to one \textit{.ucf} file. Comments begin with \#
and may appear anywhere outside strings.



\section{Mandatory sections}

\begin{spectable}{Section}{Purpose}
  \spectablerow{[ucf-meta]}{Meta-information about file version, generator, timestamps}
  \spectablerow{[complainant-details]}{Citizen identity and contact details}
  \spectablerow{[complaint-details]}{Core complaint description and classification}
  \spectablerow{[location]}{Location of the issue (as coordinates or link)}
  \spectablerow{[[attachment-details]]}{List of zero or more attachments (image, video, etc.)}
  \spectablerow{[complaint-status]}{Complaint state and optional status notes}
\end{spectable}

\section{Mandatory sections descriptions}

\subsection{[ucf-meta]}

Metadata describing this UCF file.

\begin{specsubtable}{Key}{Type}{Required}{Description}
  \specsubtablerow{version}{string}{yes}{UCF specification version (e.g., "0.2")}
  \specsubtablerow{generated-at}{datetime (RFC 3339)}{yes}{UTC or local datetime of file creation}
  \specsubtablerow{source-app}{string}{yes}{Name or identifier of the application that generated this file}
\end{specsubtable}

Example:
\medskip
\begin{lstlisting}[language=bash]
[ucf-meta]
version = "0.2"
generated-at = "2025-10-22T17:30:00+05:30"
source-app = "MyCitizenApp"
\end{lstlisting}

\subsection{[complainant-details]}

Basic details about the complainant (sometimes auto-populated).

\begin{specsubtable}{Key}{Type}{Required}{Description}
  \specsubtablerow{name}{string}{yes}{Full name of complainant}
  \specsubtablerow{contact}{string}{yes}{Primary contact number}
  \specsubtablerow{email}{string}{no}{Optional email address}
  \specsubtablerow{auth-id}{string}{no}{Unique identifier (ID from authentication system)}
\end{specsubtable}

Example:
\medskip
\begin{lstlisting}[language=bash]
[complainant-details]
name = "ABC Kumar"
contact = "1234567890"
email = "abc@example.com"
auth-id = "citizen_02349"
\end{lstlisting}

\subsection{[complaint-details]}

Core description of the issue and the department responsible.

\begin{specsubtable}{Key}{Type}{Required}{Description}
  \specsubtablerow{description}{string}{yes}{Human-readable complaint text}
  \specsubtablerow{category}{string}{no}{Complaint theme (sanitation, traffic, etc.)}
  \specsubtablerow{department}{string}{yes}{Target department identifier}
  \specsubtablerow{priority}{string}{no}{One of low, normal, high}
  \specsubtablerow{submission-method}{string}{yes}{Originating platform: app, web, kiosk, helpline, etc.}
  \specsubtablerow{attachment}{boolean}{yes}{Whether this complaint has one or more attachments}
  \specsubtablerow{ticket-id}{string}{no}{assigned id by the generation system}
\end{specsubtable}

Example:
\medskip
\begin{lstlisting}[language=bash]
[complaint-details]
description = "Garbage not cleaned for 3 days"
category = "sanitation"
department = "bbmp"
priority = "normal"
submission-method = "app"
attachment = true
\end{lstlisting}

\subsection{[location]}

Defines the place related to the complaint.

\begin{specsubtable}{Key}{Type}{Required}{Description}
  \specsubtablerow{method}{string}{yes}{How the location is represented: url or coordinates}
  \specsubtablerow{url}{string}{conditional}{Required if method = "url"}
  \specsubtablerow{latitude}{float}{conditional}{Required if method = "coordinates"}
  \specsubtablerow{longitude}{float}{conditional}{Required if method = "coordinates"}
\end{specsubtable}

Example:
\medskip
\begin{lstlisting}[language=bash]
[location]
method = "coordinates"
latitude = 13.00753
longitude = 77.65592
\end{lstlisting}

\subsection{[[attachment-details]]}

An array of tables, each describing a single attachment.
If no attachments exist, this section may be omitted.

\begin{specsubtable}{Key}{Type}{Required}{Description}
  \specsubtablerow{url}{string}{yes}{Publicly accessible or signed media link}
  \specsubtablerow{file-type}{string}{yes}{MIME type (image/jpeg, video/mp4, etc.)}
  \specsubtablerow{description}{string}{no}{Short textual note about attachment}
  \specsubtablerow{hash}{string}{no}{MD5 or SHA256 hash for integrity verification}
\end{specsubtable}

Example:
\medskip
\begin{lstlisting}[language=bash]
[[attachment-details]]
url = "https://photos.app.goo.gl/4MsLGvJeGZoWUyL67"
file-type = "image/jpeg"
description = "Garbage pile photo"
hash = "b1946ac92492d2347c6235b4d2611184"

[[attachment-details]]
url = "https://audio.example.org/complaint-note.mp3"
file-type = "audio/mpeg"
description = "Voice note describing complaint"
hash = "ad0234829205b9033196ba818f7a872b"
\end{lstlisting}

\subsection{[complaint-status]}

Describes current status of the complaint process.

\begin{specsubtable}{Key}{Type}{Required}{Description}
  \specsubtablerow{status}{string}{yes}{One of: submitted, in-process, closed}
  \specsubtablerow{comment}{string}{no}{Optional remarks from operator or system}
  \specsubtablerow{last-updated}{datetime}{no}{Last modification timestamp}
  \specsubtablerow{tracker-link}{string}{no}{External link to track live progress}
\end{specsubtable}

Example:
\medskip
\begin{lstlisting}[language=bash]
[complaint-status]
status = "submitted"
comments = ""
last-updated = "2025-10-22T17:32:00+05:30"
tracker-link = ""
\end{lstlisting}

\section{Optional Extensions}

Departments or applications may define additional sections using custom tables,
each new table should use either a prefix naming scheme ([custom.<name>]) or
include a namespace hint to avoid collision.


% TODO: wrap this in a box

 \begin{minted}[bgcolor=black!5]{toml}

 # ========================
 # UNIVERSAL COMPLAINT FILE
 # ========================

 [ucf-meta]
 version = "0.1"
 generated-at = "2025-10-22T17:20:00+05:30"
 source-app = "MyCitizenApp"

 [complainant-details]
 name = "ABC Kumar"
 contact = "1234567890"
 email = "abc@example.com"        # optional
 auth-id = "citizen_02349"        # optional unique user identifier

 [complaint-details]
 description = "Garbage not cleaned for 3 days"
 category = "sanitation"
 department = "bbmp"
 priority = "high"                # low/normal/high
 submission-method = "app"        # web/app/kiosk/helpline
 attachment = true
 related-ticket-id = ""           # optional, for linked submissions

 [location]
 method = "url"                    # url / coordinates
 url = "https://maps.app.goo.gl/8fwXBzzBW4oocjth7"
 latitude = "13.00753"
 longitude = "77.65592"

 [[attachment-details]]
 url = "https://photos.app.goo.gl/4MsLGvJeGZoWUyL67"
 file-type = "image/jpeg"
 description = "Photo of garbage dump"
 hash = "b1946ac92492d2347c6235b4d2611184"

 [complaint-status]
 status = "submitted"              # submitted / in-process / closed
 latest-comments = ""
 last-updated = "2025-10-22T17:25:00+05:30"
 tracker-link = ""                 # optional, if public tracking available
 
 \end{minted}


\section{Compliance}

A file is considered UCF-compliant if:

\begin{itemize}
  \item It conforms to TOML v1.0.0.
  \item It includes the mandatory sections and required keys.
  \item Datetime should be RFC 3339 formatted.
  \item Booleans should be lowercase true/false.
  \item Optional sections, if present, respect type and naming rules.
  \item Arrays: Only [[]] are allowed for repeated structures (attachments)
\end{itemize}

\section{FAQs}

\begin{specpara}{Why TOML?}
\href{https://toml.io/en}{TOML} is human-readable, simple, and supported across many
programming \href{https://github.com/toml-lang/toml/wiki}{languages}. Its 
well-defined syntax ensures minimal ambiguity, making UCF files easy to parse, 
validate, and generate.
\end{specpara}

\begin{specpara}{Which fields will be autopopulated?}
Fields like `name`, `contact`, `location`, and `auth-id` from user profiles and
device metadata.
\end{specpara}

\begin{specpara}{How blockchain helps here?}
A hash (e.g., SHA256 of the entire `.ucf` file) can be embedded or stored
separately in on-chain complaint registries for verification.
\end{specpara}

\begin{specpara}{Is it extensible?}
Departments can introduce new optional tables
(e.g., `[complaint-meta]`, `[response-records]`) without breaking
compatibility.
\end{specpara}

\begin{specpara}{Is it interoperable?}
Because TOML is widely supported, implementations in various languages (Python,
Go, Rust, Node.js, etc.) can easily parse or generate `.ucf` files. So
multiple system implementations can work easily with the same format.
\end{specpara}

\begin{specpara}{Is privacy a concern here?}
Sensitive citizen information (e.g., name, contact numbers) would be encrypted.
The plain text citizen info shown above is just for ease of readability.
\end{specpara}

\end{document}
