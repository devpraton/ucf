
\documentclass[12pt]{spec}

\usepackage{minted}

\begin{document}

\specheader
\bigskip
\hrule

\spectitle{Universal Complaint Format Specification}{VERSION: v0.1}

\specjust{A standardized open-source TOML-based format for registering generic
complaints against public or private entities. This format
is \textbf{citizen-facing} and focuses on interoperability across different
complaint systems, departments, and municipalities.}

\specnote{It is \textbf{not} designed for internal workflow or tracking by public departments.}


\section{General structure}

A \textit{.ucf} file is a valid TOML document in `UTF-8` encoding, containing
multiple table sections. Each section describes a logical part of the
complaint.

Each complaint corresponds to one \textit{.ucf} file. Comments begin with \#
and may appear anywhere outside strings.


\section{Mandatory sections}
\begin{spectable}{Section}{Purpose}
  \spectabelrow{[ucf-meta]}{Meta-information about file version, generator, timestamps}
  \spectabelrow{[complainant-details]}{Citizen identity and contact details}
  \spectabelrow{[complaint-details]}{Core complaint description and classification}
  \spectabelrow{[location]}{Location of the issue (as coordinates or link)}
  \spectabelrow{[[attachment-details]]}{List of zero or more attachments (image, video, etc.)}
  \spectabelrow{[complaint-status]}{Complaint state and optional status notes}
\end{spectable}

\newpage

 \begin{minted}[bgcolor=black!5]{toml}

 # ========================
 # UNIVERSAL COMPLAINT FILE
 # ========================

 [ucf-meta]
 version = "0.1"
 generated-at = "2025-10-22T17:20:00+05:30"
 source-app = "MyCitizenApp"

 [complainant-details]
 name = "ABC Kumar"
 contact = "1234567890"
 email = "abc@example.com"        # optional
 auth-id = "citizen_02349"        # optional unique user identifier

 [complaint-details]
 description = "Garbage not cleaned for 3 days"
 category = "sanitation"
 department = "bbmp"
 priority = "high"                # low/normal/high
 submission-method = "app"        # web/app/kiosk/helpline
 attachment = true
 related-ticket-id = ""           # optional, for linked submissions

 [location]
 method = "url"                    # url / coordinates
 url = "https://maps.app.goo.gl/8fwXBzzBW4oocjth7"
 latitude = "13.00753"
 longitude = "77.65592"

 [[attachment-details]]
 url = "https://photos.app.goo.gl/4MsLGvJeGZoWUyL67"
 file-type = "image/jpeg"
 description = "Photo of garbage dump"
 hash = "b1946ac92492d2347c6235b4d2611184"

 [complaint-status]
 status = "submitted"              # submitted / in-process / closed
 latest-comments = ""
 last-updated = "2025-10-22T17:25:00+05:30"
 tracker-link = ""                 # optional, if public tracking available
 
 \end{minted}


\section{[ucf-meta]}

Metadata describing this UCF file.

\begin{specpara}{Why TOML?}

\href{https://toml.io/en}{TOML} is human-readable, simple, and supported across many
programming \href{https://github.com/toml-lang/toml/wiki}{languages}. Its 
well-defined syntax ensures minimal ambiguity, making UCF files easy to parse, 
validate, and generate.

\end{specpara}

\begin{specitemize}{Additional Notes}

  \item Autopopulation:  Apps can automatically fill fields
like `name`, `contact`, `location`, and `auth-id` from user profiles and device
metadata.

  \item Hashing / Blockchain:  A hash (e.g., SHA256 of the entire `.ucf` file)
can be embedded or stored separately in on-chain complaint registries for
verification.

  \item Extensibility:  Departments can introduce new optional tables
(e.g., `[complaint-meta]`, `[response-records]`) without breaking
compatibility.

  \item Interoperability:  Because TOML is widely supported, implementations in
various languages (Python, Go, Rust, Node.js, etc.) can easily parse or
generate `.ucf` files.

  \item Privacy Considerations:  Sensitive citizen information (e.g., contact
numbers, precise coordinates) should be encrypted or excluded from public
blockchain entries.

\end{specitemize}

\end{document}
