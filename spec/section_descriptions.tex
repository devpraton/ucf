
\section{Mandatory sections}

\begin{spectable}{Section}{Purpose}
  \spectablerow{[ucf-meta]}{Meta-information about file version, generator, timestamps}
  \spectablerow{[complainant-details]}{Citizen identity and contact details}
  \spectablerow{[complaint-details]}{Core complaint description and classification}
  \spectablerow{[location]}{Location of the issue (as coordinates or link)}
  \spectablerow{[[attachment-details]]}{List of zero or more attachments (image, video, etc.)}
  \spectablerow{[complaint-status]}{Complaint state and optional status notes}
\end{spectable}

\section{Mandatory sections descriptions}

\subsection{[ucf-meta]}

Metadata describing this UCF file.

\begin{specsubtable}{Key}{Type}{Required}{Description}
  \specsubtablerow{version}{string}{yes}{UCF specification version (e.g., "0.2")}
  \specsubtablerow{generated-at}{datetime (RFC 3339)}{yes}{UTC or local datetime of file creation}
  \specsubtablerow{source-app}{string}{yes}{Name or identifier of the application that generated this file}
\end{specsubtable}

Example:
\medskip
\begin{lstlisting}[language=bash]
[ucf-meta]
version = "0.2"
generated-at = "2025-10-22T17:30:00+05:30"
source-app = "MyCitizenApp"
\end{lstlisting}

\subsection{[complainant-details]}

Basic details about the complainant (sometimes auto-populated).

\begin{specsubtable}{Key}{Type}{Required}{Description}
  \specsubtablerow{name}{string}{yes}{Full name of complainant}
  \specsubtablerow{contact}{string}{yes}{Primary contact number}
  \specsubtablerow{email}{string}{no}{Optional email address}
  \specsubtablerow{auth-id}{string}{no}{Unique identifier (ID from authentication system)}
\end{specsubtable}

Example:
\medskip
\begin{lstlisting}[language=bash]
[complainant-details]
name = "ABC Kumar"
contact = "1234567890"
email = "abc@example.com"
auth-id = "citizen_02349"
\end{lstlisting}

\subsection{[complaint-details]}

Core description of the issue and the department responsible.

\begin{specsubtable}{Key}{Type}{Required}{Description}
  \specsubtablerow{description}{string}{yes}{Human-readable complaint text}
  \specsubtablerow{category}{string}{no}{Complaint theme (sanitation, traffic, etc.)}
  \specsubtablerow{department}{string}{yes}{Target department identifier}
  \specsubtablerow{priority}{string}{no}{One of low, normal, high}
  \specsubtablerow{submission-method}{string}{yes}{Originating platform: app, web, kiosk, helpline, etc.}
  \specsubtablerow{attachment}{boolean}{yes}{Whether this complaint has one or more attachments}
  \specsubtablerow{ticket-id}{string}{no}{assigned id by the generation system}
\end{specsubtable}

Example:
\medskip
\begin{lstlisting}[language=bash]
[complaint-details]
description = "Garbage not cleaned for 3 days"
category = "sanitation"
department = "bbmp"
priority = "normal"
submission-method = "app"
attachment = true
\end{lstlisting}

\subsection{[location]}

Defines the place related to the complaint.

\begin{specsubtable}{Key}{Type}{Required}{Description}
  \specsubtablerow{method}{string}{yes}{How the location is represented: url or coordinates}
  \specsubtablerow{url}{string}{conditional}{Required if method = "url"}
  \specsubtablerow{latitude}{float}{conditional}{Required if method = "coordinates"}
  \specsubtablerow{longitude}{float}{conditional}{Required if method = "coordinates"}
\end{specsubtable}

Example:
\medskip
\begin{lstlisting}[language=bash]
[location]
method = "coordinates"
latitude = 13.00753
longitude = 77.65592
\end{lstlisting}

\subsection{[[attachment-details]]}

An array of tables, each describing a single attachment.
If no attachments exist, this section may be omitted.

\begin{specsubtable}{Key}{Type}{Required}{Description}
  \specsubtablerow{url}{string}{yes}{Publicly accessible or signed media link}
  \specsubtablerow{file-type}{string}{yes}{MIME type (image/jpeg, video/mp4, etc.)}
  \specsubtablerow{description}{string}{no}{Short textual note about attachment}
  \specsubtablerow{hash}{string}{no}{MD5 or SHA256 hash for integrity verification}
\end{specsubtable}

Example:
\medskip
\begin{lstlisting}[language=bash]
[[attachment-details]]
url = "https://photos.app.goo.gl/4MsLGvJeGZoWUyL67"
file-type = "image/jpeg"
description = "Garbage pile photo"
hash = "b1946ac92492d2347c6235b4d2611184"

[[attachment-details]]
url = "https://audio.example.org/complaint-note.mp3"
file-type = "audio/mpeg"
description = "Voice note describing complaint"
hash = "ad0234829205b9033196ba818f7a872b"
\end{lstlisting}

\subsection{[complaint-status]}

Describes current status of the complaint process.

\begin{specsubtable}{Key}{Type}{Required}{Description}
  \specsubtablerow{status}{string}{yes}{One of: submitted, in-process, closed}
  \specsubtablerow{comment}{string}{no}{Optional remarks from operator or system}
  \specsubtablerow{last-updated}{datetime}{no}{Last modification timestamp}
  \specsubtablerow{tracker-link}{string}{no}{External link to track live progress}
\end{specsubtable}

Example:
\medskip
\begin{lstlisting}[language=bash]
[complaint-status]
status = "submitted"
comments = ""
last-updated = "2025-10-22T17:32:00+05:30"
tracker-link = ""
\end{lstlisting}

\section{Optional Extensions}

Departments or applications may define additional sections using custom tables,
each new table should use either a prefix naming scheme ([custom.<name>]) or
include a namespace hint to avoid collision.
